\documentclass[14pt]{beamer}
\usetheme{Antibes}
\definecolor{myOrange}{RGB}{238,97,53}
\setbeamercolor*{palette primary}{fg=myOrange}
\setbeamercolor*{palette secondary}{fg=myOrange,bg=white}
\setbeamercolor*{palette tertiary}{bg=myOrange,fg=white}
\setbeamercolor*{titlelike}{parent=palette primary}
\setbeamercolor*{itemize item}{fg=myOrange}
\setbeamercolor{block title example}{bg=myOrange} 
\setbeamercolor{section in toc}{fg=black}
\setbeamertemplate{sections/subsections in toc}[sections numbered]
\setbeamercolor{section number projected}{bg=myOrange,fg=yellow}
\setbeamertemplate{footline}[frame number]
\setbeamertemplate{frametitle}{
  \begin{centering}
  \insertframetitle
    \par
    \end{centering}
}
\setbeamertemplate{itemize item}{\textbullet}

\usepackage{amsmath}
\usepackage{ulem}
\usepackage{alltt}
\usepackage[english]{babel}
\usepackage{listings}
\usepackage{graphicx}
\definecolor{javared}{rgb}{0.6,0,0} % for strings
\definecolor{javagreen}{rgb}{0.25,0.5,0.35} % comments
\definecolor{javapurple}{rgb}{0.5,0,0.35} % keywords
\definecolor{javadocblue}{rgb}{0.25,0.35,0.75} % javadoc
\definecolor{javalinenumbers}{RGB}{79,80,81} % linenumbers
\definecolor{grey}{rgb}{0.95,0.95,0.95}
\lstset{language=Java,
basicstyle=\ttfamily \footnotesize,
keywordstyle=\color{myOrange}\textbf,
stringstyle=\color{javared},
commentstyle=\color{javagreen},
morecomment=[s][\color{javadocblue}]{/**}{*/},
numberstyle=\color{javalinenumbers}\tiny,
numbersep=10pt,
tabsize=4,
showspaces=false,
showstringspaces=false,
breaklines=true,
firstnumber=100,
backgroundcolor=\color{grey},
emph={
  val, var, def,
},
emphstyle=\color{myOrange}\bfseries,
morecomment=[s][\color{blue}]{@}{\ }
}


\usepackage{hyperref}
\logo{\includegraphics[width=1cm]{logo.png}}
\newcommand\B{\rule[-1.7ex]{0pt}{0pt}}
\def\colored#1{\textcolor{myOrange}{#1}}

\setcounter{tocdepth}{1}

\begin{document}
\title{Testing Like A Boss}
\author{Arkady Galyash}
\institute{TosChart}
\date{\today}

\newcommand{\smaller}[1] {
  {\scriptsize {#1}}
}

% Здравствуйте, меня зовут Аркадий Галяш, я работаю в команде TosChart. 
% Сегодня я представляю вашему вниманию презентацию на тему "Java со ввкусом огурца", посвященную проблемам и преимуществам использования инструментария BDD в реальных проектах
% Вопросы можно задавать как походу доклада, так и в конце.
\frame{\titlepage}

\frame%
{\frametitle{Agenda}
  \tableofcontents[1]
}

\section{How regular JUnit test looks like (in time)?}
\subsection{Intro}
\frame{\frametitle{Intro}
  \begin{itemize}
    \item John Doe The Programmer
    \item Java developer @ Moon Ms
    \item Since 2000
  \end{itemize}
}

\subsection{June 2000}
\frame{\frametitle{June 2000}
  \begin{center}
    \includegraphics[width=0.4\textwidth]{2000.png} \\
    Just graduated, "tests are for chicken"
  \end{center}
}

\subsection{May 2001}
\frame{\frametitle{May 2001}
  \begin{center}
    \includegraphics[width=0.4\textwidth]{2001.png} \\
    User reports a bug, JUnit discovered
  \end{center}
}


\begin{frame}[fragile]
\frametitle{May 2001}
\begin{center}
\begin{lstlisting}[frame=single]
public void testSimpleAdd() {
    Money m12 = new Money(12, "USD");
    Money m14 = new Money(14, "USD");
    Money expected = new Money(26, "USD");
    Money result = m12.add(m14);
    assert (expected.equals(result));
}
\end{lstlisting} 
\end{center}
\end{frame}

\subsection{July 2006}
\frame{\frametitle{July 2006}
  \begin{center}
    \includegraphics[width=0.4\textwidth]{2006.png} \\
    User reports another bug, migrate to JUnit 4
  \end{center}
}


\begin{frame}[fragile]
\frametitle{July 2006}
\begin{center}
\begin{lstlisting}[frame=single]
@Test public void simpleAdd() {
  ...
}
@Test public void addNull() {
    Money m12 = new Money(12, "USD");
    Money result = m12.add(null);
    assertTrue(m12.equals(result));
}
\end{lstlisting} 
\end{center}
\end{frame}

\subsection{December 2007}
\frame{\frametitle{December 2007}
  \begin{center}
    \includegraphics[width=0.4\textwidth]{2007.png} \\
    User reports one more bug
  \end{center}
}


\begin{frame}[fragile]
\frametitle{December 2007}
\begin{center}
\begin{lstlisting}[frame=single]
@Test public void simpleAdd() {
  ...
}
@Test public void addNull() {
  ...
}
@Test public void addBillions() {
    Money max = new Money(MAX_VALUE, "USD");
    Money one = new Money(1, "USD");
    Money result = max.add(one);
    assertTrue(result.getAmount() > 0);
}
\end{lstlisting} 
\end{center}
\end{frame}

\subsection{October 2009}
\frame{\frametitle{October 2009}
  \begin{center}
    \includegraphics[width=0.4\textwidth]{2009.png} \\
    Time to refactoring
  \end{center}
}


\begin{frame}[fragile]
\frametitle{October 2009}
\begin{center}
\begin{lstlisting}[frame=single]
@Test public void simpleAdd(); 
@Test public void addNull();
@Test public void addBillions();

private void testAdd(Money _1, Money _2) {
  Money result = _1.add(_2);
  assertEquals(result.getAmount(), 
         _1.getAmount() + _2.getAmount());
}
\end{lstlisting} 
\end{center}
\end{frame}

\subsection{Long way}
\frame{\frametitle{Long way}
  \begin{center}
    \includegraphics[width=0.27\textwidth]{2000.png} 
    \includegraphics[width=0.27\textwidth]{2001.png} \\
    \includegraphics[width=0.27\textwidth]{2006.png} 
    \includegraphics[width=0.27\textwidth]{2007.png} 
    \includegraphics[width=0.27\textwidth]{2009.png} \\
  \end{center}
}

\frame{\frametitle{Long way}
  \begin{itemize}
    \item Figure out some test inputs
    \item Code duplication
    \item Understand post- and pre-conditions
    \item Separate creating test inputs from assertions 
  \end{itemize}
}

% Ссылки 
\section{Links}
\frame{\frametitle{Links}
  \begin{itemize}
    \item \href{http://www.haskell.org/haskellwiki/Introduction_to_QuickCheck2}{Introduction to QuickCheck2}
  \end{itemize}
}
\frame{
  \begin{center}
    Thank You!
  \end{center}
}

\end{document}
