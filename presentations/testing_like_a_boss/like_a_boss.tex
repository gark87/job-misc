\documentclass[14pt]{beamer}
\usetheme{Antibes}
\definecolor{myOrange}{RGB}{238,97,53}
\setbeamercolor*{palette primary}{fg=myOrange}
\setbeamercolor*{palette secondary}{fg=myOrange,bg=white}
\setbeamercolor*{palette tertiary}{bg=myOrange,fg=white}
\setbeamercolor*{titlelike}{parent=palette primary}
\setbeamercolor*{itemize item}{fg=myOrange}
\setbeamercolor{block title example}{bg=myOrange} 
\setbeamercolor{section in toc}{fg=black}
\setbeamertemplate{sections/subsections in toc}[sections numbered]
\setbeamercolor{section number projected}{bg=myOrange,fg=yellow}
\setbeamertemplate{footline}[frame number]
\setbeamertemplate{frametitle}{
  \begin{centering}
  \insertframetitle
    \par
    \end{centering}
}
\setbeamertemplate{itemize item}{\textbullet}

\usepackage{amsmath}
\usepackage{ulem}
\usepackage{alltt}
\usepackage[english]{babel}
\usepackage{listings}
\usepackage{graphicx}
\definecolor{javared}{rgb}{0.6,0,0} % for strings
\definecolor{javagreen}{rgb}{0.25,0.5,0.35} % comments
\definecolor{javapurple}{rgb}{0.5,0,0.35} % keywords
\definecolor{javadocblue}{rgb}{0.25,0.35,0.75} % javadoc
\definecolor{javalinenumbers}{RGB}{79,80,81} % linenumbers
\definecolor{grey}{rgb}{0.95,0.95,0.95}
\lstset{language=Java,
basicstyle=\ttfamily \footnotesize,
keywordstyle=\color{myOrange}\textbf,
stringstyle=\color{javared},
commentstyle=\color{javagreen},
morecomment=[s][\color{javadocblue}]{/**}{*/},
numberstyle=\color{javalinenumbers}\tiny,
numbersep=10pt,
tabsize=4,
showspaces=false,
showstringspaces=false,
breaklines=true,
firstnumber=100,
backgroundcolor=\color{grey},
emph={
  val, var, def,
},
emphstyle=\color{myOrange}\bfseries,
morecomment=[s][\color{blue}]{@}{\ }
}


\usepackage{hyperref}
\logo{\includegraphics[width=1cm]{logo.png}}
\newcommand\B{\rule[-1.7ex]{0pt}{0pt}}
\def\colored#1{\textcolor{myOrange}{#1}}

\setcounter{tocdepth}{1}

\begin{document}
\title{Testing Like A Boss}
\author{Arkady Galyash}
\institute{TosChart}
\date{\today}

\newcommand{\smaller}[1] {
  {\scriptsize {#1}}
}

% Здравствуйте, меня зовут Аркадий Галяш, я работаю в команде TosChart. 
% Сегодня я представляю вашему вниманию презентацию на тему "Java со ввкусом огурца", посвященную проблемам и преимуществам использования инструментария BDD в реальных проектах
% Вопросы можно задавать как походу доклада, так и в конце.
\frame{\titlepage}

\frame%
{\frametitle{Agenda}
  \tableofcontents[1]
}

\section{How regular JUnit test looks like (in time)?}
\subsection{Intro}
\frame{\frametitle{Intro}
  \begin{itemize}
    \item John Doe The Programmer
    \item Java developer @ Moon Ms
    \item Since 2000
  \end{itemize}
}

\subsection{June 2000}
\frame{\frametitle{June 2000}
  \begin{center}
    \includegraphics[width=0.4\textwidth]{2000.png} \\
    Just graduated, "tests are for chicken"
  \end{center}
}

\subsection{May 2001}
\frame{\frametitle{May 2001}
  \begin{center}
    \includegraphics[width=0.4\textwidth]{2001.png} \\
    JUnit discovered
  \end{center}
}

\subsection{July 2006}
\frame{\frametitle{July 2006}
  \begin{center}
    \includegraphics[width=0.4\textwidth]{2006.png} \\
    User tries to search non-existant element
  \end{center}
}

\subsection{December 2007}
\frame{\frametitle{December 2007}
  \begin{center}
    \includegraphics[width=0.4\textwidth]{2007.png} \\
    User tries to search in array with duplicates
  \end{center}
}

\subsection{October 2009}
\frame{\frametitle{October 2009}
  \begin{center}
    \includegraphics[width=0.4\textwidth]{2009.png} \\
    Integer overflow
  \end{center}
}

\subsection{Summary}
\begin{frame}[t]
  \frametitle{Long way}
  \begin{center}
  \begin{tabular}{c @{} c @{} c @{} c @{} c}
    \includegraphics[width=0.2\textwidth]{2000.png} &
    \includegraphics[width=0.2\textwidth]{2001.png} &
    \includegraphics[width=0.2\textwidth]{2006.png} &
    \includegraphics[width=0.2\textwidth]{2007.png} &
    \includegraphics[width=0.2\textwidth]{2009.png} \\
    2000 & 2001 & 2006 & 2007 & 2009
  \end{tabular}
  \end{center}
\end{frame}

\frame{\frametitle{xUnit}
  \begin{itemize}
    \item We need to think about test cases
    \item Until we've discover one, we cannot find a bug
    \item If we discover test case, this test is useless right now, it may be useful only in future
  \end{itemize}
}

\frame{\frametitle{QuickCheck}
  \begin{itemize}
    \item Another approach
    \item 
    \item If we discover test case, this test is useless right now, it may be useful only in future
  \end{itemize}
}

% Ссылки 
\section{Links}
\frame{\frametitle{Links}
  \begin{itemize}
    \item \href{http://www.haskell.org/haskellwiki/Introduction_to_QuickCheck2}{Introduction to QuickCheck2}
  \end{itemize}
}
\frame{
  \begin{center}
    Thank You!
  \end{center}
}

\end{document}
